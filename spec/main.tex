%!TEX root = ./main.tex
%!TEX program = xelatex

\documentclass[12pt]{article}

\usepackage{hyperref}
\usepackage{inputenc}
\usepackage{parskip}
\usepackage{geometry}
\usepackage{bookmark}
\usepackage{titlesec}
\usepackage[main=british,polish]{babel}
\usepackage{datetime2}
\usepackage[protrusion, expansion, tracking]{microtype}
\usepackage[type={CC}, modifier={by-nc-sa}, version={4.0}]{doclicense}

\let\oldparagraph\paragraph
\renewcommand{\paragraph}[1]{\oldparagraph{#1}\mbox{}\\}

\geometry{a4paper, margin=1.75cm}
\setcounter{secnumdepth}{4}
\setcounter{tocdepth}{4}
\hypersetup{
    colorlinks=true,
    linkcolor=black,
    filecolor=black,
    urlcolor=blue,
}
\bibliographystyle{plain}

\title{%
    Papyrus (Like) Configuration Language Specification \\
    \large Version CURRENT
}
\author{\foreignlanguage{polish}{Natalia Łotocka} - \url{https://natawie.gay}}

\begin{document}
    \maketitle
    \tableofcontents
    \newpage

    \section{Introduction}
    \subsection{Why is this?}
    One day coming back from school I had a thought: "A web server that for some reason runs in Skyrim". \\
    So like the masochistic \& sadistic little gremlin that I am, I started working on it. \\
    I got to the point where I had an extremely simple HTTP/1.1 (only) server (it only supported GET and HEAD requests), but now I wanted to have a way of configuring it. \\
    My first thought was to use something... normal? sensible? logical? But then I remembered what I was working on and thought nothing like that would fit the \textbf{vibe}. \\
    For a split second I thought about using XML, but I'm not that insane, so I created something that's most definitely way worse than XML. \\
    Basically, this has absolutely no reason to exist. \\
    But it does. \\
    I'm sorry (not really).

    \subsection{Why would I use this?}
    Masochism. \\
    Hatred of oneself. \\
    Boredom. \\
    You're creating a mod for a BGS game and want to spread the "joy" of Papyrus to your users. \\
    You can tell this is \href{https://github.com/natawie/PapyrusLikeConfigurationLanguageToHTML}{the future of web development}.

    \subsection{Why does this specification exist?}
    No reason. \\
    I was just bored. \\
    Also, I don't know when to end a bit.

    \newpage

    \section{Syntax}
    \subsection{Shared}

    \subsubsection{General}
    All keywords are case-insensitive. \\
    Indentation is ignored.

    \subsubsection{Comments}
    Comments start with a semicolon and end at the end of the line.

    \subsubsection{Types}
    There are currently 3 supported types: strings, integers, and booleans. \\
    For templates, their keywords are \texttt{string}, \texttt{int}, and \texttt{bool} respectively.

    \subsubsection{Values}
    Parsers are expected to error out if a value is of the wrong type.
   
    \paragraph{Strings}
    Strings are enclosed in double quotes. \\
    They can contain anything except for a double quote unless escaped with a backslash.

    \paragraph{Integers}
    Numbers up to the 64-bit signed integer limit are supported.

    \paragraph{Booleans}
    Booleans are represented by the \texttt{true} and \texttt{false} keywords.

    \subsubsection{Names}
    Names are used to identify attributes and lists. \\
    They're case-insensitive and have to start with either an underscore or a letter, but can contain numbers as well.

    \newpage

    \subsection{Configuration Only}

    \subsubsection{Configuration Attributes}
    Configuration attributes are key-value pairs. \\
    They're defined by a name, an equals sign, and a value. \\
    They can be of any type.

    \subsubsection{Imports}
    Imports are only for possible dynamic error-checking or autocompletion. \\
    They're defined by the \texttt{Import} keyword followed by a string value, either a path or a URL to a template.

    \subsubsection{Configuration Elements}
    Elements are akin to objects or structs in other languages. \\
    They're defined by the \texttt{ConfigElement} keyword followed by a name, an optional list of configuration attributes, an optional list of other elements, and the \texttt{EndConfigElement} keyword. \\
    Attributes are written in the main body of the element, unlike elements which need to be written in a \texttt{ConfigList}.

    \subsubsection{Configuration Lists}
    Lists are a collection of elements. \\
    Each \texttt{ConfigElement} has to be wrapped in a \texttt{ConfigListElement}/\texttt{EndConfigListElement} block. \\
    Each \texttt{ConfigListElement} has to have an ID - an integer value - after the first keyword. \\
    The IDs have to be unique, sequential, and start from 0. \\
    A \texttt{ConfigListElement} block can only contain one element. \\
    Lists are defined by the \texttt{ConfigList} keyword followed by a name, an optional list of \texttt{ConfigElement}s in \texttt{ConfigListElement} blocks, and the \texttt{EndConfigList} keyword.

    \subsubsection{Configuration}
    A Configuration is a collection of \texttt{ConfigElement}s and \texttt{ConfigList}s. \\
    It starts with the \texttt{ConfigName} keyword followed by a name. \\
    It can contain multiple imports, elements, and lists.

    \newpage

    \subsection{Templates Only}

    \subsubsection{Options}
    Options are literally just configuration attributes. \\
    The only difference is that they're written in a special block. \\
    The options that are currently supported are: \texttt{required}, \texttt{minimumCount}, and \texttt{maximumCount}. \\
    \texttt{required} is a boolean, \texttt{minimumCount} and \texttt{maximumCount} are integers and they only apply to lists. \\
    The block is defined by either the \texttt{TemplateElementOptions} or the \texttt{TemplateListOptions} keyword depending on where it's used, a list of options, and either the \\ \texttt{EndTemplateElementOptions} or the \texttt{EndTemplateListOptions} keyword respectively.

    \subsubsection{Template Attributes}
    Template attributes are type declarations of configuration attributes. \\
    They're defined by a type, a name, an optional required keyword, and an optional default keyword followed by a value.

    \subsubsection{Template Elements}
    Elements are akin to objects or structs in other languages. \\
    They're defined by the \texttt{TemplateElement} keyword followed by a name, an optional options block, an optional list of template attributes, an optional list of other elements, and the \texttt{EndTemplateElement} keyword. \\
    Attributes are written in the main body of the element, unlike elements which need to be written in a \texttt{TemplateList}.

    \subsubsection{Template Lists}
    Lists are a collection of elements. \\
    Each \texttt{TemplateElement} has to be wrapped in a \texttt{TemplateListElement}/\texttt{EndTemplateListElement} block. \\
    Each \texttt{TemplateListElement} has to have an ID - an integer value - after the first keyword. \\
    The IDs have to be unique, and at least one \texttt{TemplateListElement} has to have an ID of 0. \\
    If a \texttt{ConfigListElement} has an ID that's not defined in a \texttt{TemplateList}, it takes the attributes of the element with the ID closest to it that's lower than it. \\
    A \texttt{TemplateListElement} block can only contain one \texttt{TemplateElement}. \\
    Lists are defined by the \texttt{TemplateList} keyword followed by a name, an optional list of \\ \texttt{TemplateElement}s in \texttt{TemplateListElement} blocks, and the \texttt{EndTemplateList} keyword.

    \subsubsection{Template}
    A Template is a collection of \texttt{TemplateElement}s and \texttt{TemplateList}s. \\
    It starts with the \texttt{TemplateName} keyword followed by a name, and after that, it can contain multiple elements and lists.

    \newpage

    \section{Example}
    \begin{minipage}[t]{0.5\textwidth}
        \subsection{Configuration}
        \verbatiminput{../examples/simple.p(l)cl}
    \end{minipage}
    \begin{minipage}[t]{0.5\textwidth}
        \subsection{Template}
        \verbatiminput{../examples/simple.p(l)cltem}
    \end{minipage}
    % \subsection{Configuration}
    % \verbatiminput{../examples/simple.p(l)cl}
    % \newpage
    % \subsection{Template}
    % \verbatiminput{../examples/simple.p(l)cltem}

    \newpage

    \section{Licence}
    \doclicenseThis
\end{document}
